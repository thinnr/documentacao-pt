\chapter[Introdução]{Introdução}
\par O desenvolvimento para sistemas web, ao mesmo tempo que convidativo, pela facilidade e rapidez em ver resultados; 
pode ser intimidador, devido à quantidade de escolhas e diversas possibilidades de combinações dessas escolhas, deixando 
o aluno desencorajado ou desanimado de aprender mais. Um iniciante pode seguir o caminho do html, css e javascript, 
mas após o aprendizado dessas tecnologias, mesmo que apenas o essencial, esse estudante, pode se encontrar desamparado, 
ao querer dar o passo seguinte. Encontrando uma infinidade de frameworks, linguagens e configurações que prometem coisas 
ainda desconhecidas para ele. Pois, além de ter que aprender uma linguagem de programação, de marcação e de estilo, deverá, 
também lidar com configurações e o significado, de cada uma delas, e protocolos.
\par Desenvolvimento web requer múltiplas habilidades, que podem ser ou não natural para o desenvolvedor. Dessa forma, 
surge a necessidade de um  ambiente amigável que ajuda iniciantes a focar no programa e nas técnicas necessária para o 
desenvolvimento web.

\section{Objetivo Geral}
\par Criar um framework web, front-end que seja um ambiente amigável para estudantes. Para que o foco do aprendizado seja o 
funcionamento e boas práticas, ao invés de uma determinada tecnologia. Fazer com que o foco do estudo seja o desenvolvimento 
e facilitar o processo de configuração, aproximando iniciantes à builders, lazy load e outras técnicas, bem quistas, mas sem 
sustos para o processo de aprendizagem.

\section{Objetivo Específico}
\par Para o desenvolvimento de tal framework, deverá primeiro definir o que é fácil, o que é simples e o que é rápido. 
Para uma pessoa o paradigma funcional pode ser entendido com naturalidade, enquanto que para outro, pode ser a técnica 
mais complexa e incompreensível. Por isso se faz necessário pesquisar o que é fácil.
Da mesma forma com o conceito de simples. Simples pode ser o framework, ou o processo de desenvolver com ele, ou até ambos. 
\par Para garantir que o resultado final, atinja os resultados esperados, serão necessários, testes, pesquisas e estudo 
de casos, para acompanhamento das dificuldades, enfrentadas por diversa pessoas, com diferentes entendimento de “fácil” e  
fazer com que o framework se encaixe nessa definição. 

\subsection{Abordagem Metodológica}

\par Para que esse projeto chegue ao resultado esperado, será desenvolvido uma ferramenta de utilizando interface de linha 
de comando para interação com o framework. Que por sua vez utilizará html, css e javascript, deixando, assim, além do resultado 
o projeto em si também simples e compreensível.
