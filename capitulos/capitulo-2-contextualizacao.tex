\chapter{Fundamentação Teórica}
\label{ch:fundamentacao}
\par Para a compreensão do projeto, é necessário entender alguns conceitos teŕicos e o objetivo do projeto. Assim como as tecnologias utilizadas.

\section{Requisitos}
\subsection{Requisitos Do Sistema}
\subsubsection{Framework}
\begin{itemize}
    \item O framework deve fornecer início rápido de um projeto;
    \item O framework deve facilitar a modularização;
    \item O framework deve facilitar o roteamento dos módulos;
    \item O framework deve facilitar a implementação de \textit{lazy loading}.
\end{itemize}

\subsubsection{CLI}
\begin{itemize}
    \item O CLI deve fornecer controle simples das funcionalidades do framework;
    \item O CLI deve fornecer início rápido do projeto;
    \item O CLI deve fornecer auxílio para \textit{live development}.
\end{itemize}

\section{Tecnologias Utilizadas}

\subsection{Javascript}
\par O Javascript pode ser utilizado para finalidades e paradigmas múltiplos, adaptando-se às necessidades do desenvolvedor e do projeto.

\subsubsection{Especificações}
\textbf{Versão: } ECMAScript 2019.

\subsection{HTML5}
HTML é a linguagem de marcação, livre e open-source, aceita por todos navegadores, será utilizado para produção dos templates que serão 
renderizados no DOM.

\subsubsection{Especificações}
\textbf{Versão: } HTML5

\subsection{CSS3}
CSS é uma linguagem para modificação de estilos, muito utilizado para aplicativos web.

\subsubsection{Especificações}
\textbf{Versão: } CSS3

\subsection{Lodash}
\par Lodash é uma biblioteca de utilidades para javascript. Fornece funções e métodos para facilitar e agilizar o processo de desenvolvimento.  
Sua utilização auxilia o desenvolvimento do projeto de forma significativa. Provendo meios para que o processo de desenvolvimento fique mais 
concentrado no requisito, ao invés de ações comuns no desenvolvimento.

\subsubsection{Especificações}
\textbf{Versão: } 4.17.10

\subsection{RxJS}
\par RxJS é uma biblioteca para programação reativa. Assim como lodash, RxJS é uma biblioteca de utilidades, porém, focada em facilitar 
funções assíncronas. Sua utilização é empregada principalmente para módulos que fazem requisições HTTP ou eventos e manipulação do DOM.

\subsubsection{Especificações}
\textbf{Versão: } V6

\subsection{webpack}
\par Webpack  é um empacotador de módulos, webpack faz com que todos os arquivos do projeto, seja montado em poucos arquivos minificados, 
e prontos para produção. Sua escolha é devido a facilidade de fazer o Lazy Loading e performance quanto ao processo de desenvolvimento.

\subsubsection{Especificações}
\textbf{Versão: } 4.22.0

\subsection{CommanderJs}
CommanderJs é um framework para desenvolvimento de softwares utilizando interface de linha de comando. A escolha é devida à facilidade e rapidez no processo de desenvolvimento.

\subsubsection{Especificações}
\textbf{Versão: } v2.19.0

\subsection{NPM}
De forma básica NPM é um gerenciador de  dependências para javascript, Sua utilização é indispensável para projetos com muitas dependências 
externas e com integração contínua.

\subsubsection{Especificações}
\textbf{Versão: } 6.4.1

\subsection{Jasmine}
Framework para testes unitários, focado em Behavior-Driven Development com javascript, Sua empregação, é devido a necessidade de garantir que o software, 
funcione assim como esperado, e que novas funcionalidades sejam integradas sem afetar as já existentes.

\subsubsection{Especificações}
\textbf{Versão: } 3.2

\section{Padrões Utilizados}
\subsection{MVC}
\par Para o desenvolvimento da ferramenta de linha de comando, foi empregado o padrão de projeto MVC.  
Para a separação dos componentes responsáveis pela interação com o usuário, controle das funcionalidades e processamento das operações.
\subsection{Programação Funcional}
\par Para o desenvolvimento do framework e modelo do CLI, será empregado o paradigma funcional. Por ser apresentar uma estrutura mais 
concisa e apresentar técnicas que deixam o código mais fácil de fazer testes unitários, dessa forma deixando-o mais confiável.

